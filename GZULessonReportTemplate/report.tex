%贵州大学选修课课程报告LaTeX版(非官方)
%请使用XeLaTeX进行编译(若需要目录,请编译两次)
%版本:1.0
%作者:Spico(spico@qq.com)

\documentclass[12pt,a4paper]{ctexart} %使用CTeX文档包,正文部分字号12,A4纸张
\usepackage{ulem}   %用于字体加强
\usepackage{amsmath}    %用于数学公式
\usepackage{graphicx}   %用于插入图像
\usepackage{subfig} %用于排列图像
\usepackage{booktabs}   %用于制作三线表格
\usepackage{multirow}   %用于表格跨行
\usepackage{fancyhdr}   %用于设置页面格式
\usepackage{verbatim}   %用于打印代码(原文输出)
\usepackage{setspace}   %设置行距

\pagestyle{fancy}
\fancyhf{}
\lhead{\textbf{课程报告}}
\rhead{\thepage}
\cfoot{}
%\setCJKmainfont{宋体}
\newcommand{\Date}[3]{\heiti\zihao{2} #1年#2月#3日}    %生成日期

\begin{document}

\centering
%第一页:封面
\begin{figure}[h]
    \centering
    \subfloat{
        \label{fig:schoolBadge}
        \includegraphics[keepaspectratio,width=90pt]{pic/school_badge.png}
    }
    \hspace{10pt}
    \subfloat{
        \label{fig:schoolName}
        \includegraphics[keepaspectratio,width=200pt]{pic/school_name.png}
    }
    \newline\newline
    \centering
    \heiti\zihao{-0} 贵州大学课程报告 \newline\newline\newline
    \heiti\zihao{1} 中国酒概述---酒文化之我见 \newline\newline
\end{figure}

%生成署名信息
\begin{tabular}{rcl}
    \centering
    \zihao{2}\heiti{学\quad 院:} & \zihao{2}\uline{\makebox[9em][c]{\fangsong 大数据与信息工程}} \\ \\
    \zihao{2}\heiti{专\quad 业:} & \zihao{2}\uline{\makebox[9em][c]{\fangsong 通信工程}} \\ \\
    \zihao{2}\heiti{班\quad 级:} & \zihao{2}\uline{\makebox[9em][c]{\fangsong 通信152班}} \\ \\
    \zihao{2}\heiti{学\quad 号:} & \zihao{2}\uline{\makebox[9em][c]{\fangsong 15********}} \\ \\
    \zihao{2}\heiti{姓\quad 名:} & \zihao{2}\uline{\makebox[9em][c]{\fangsong 太子夜华}} \\ \\ \newline\newline\newline
\end{tabular}

\Date{2017}{2}{19}
%设置页面格式
\setcounter{page}{0}    %设置页码为0
\thispagestyle{empty}   %清除页眉页脚
\newpage    %强制换页

%目录页
\zihao{3}\textnormal{
\setcounter{tocdepth}{2}    %设置目录深度,生成目录
\tableofcontents}    %有目录需要编译两次
\setcounter{page}{0}
\thispagestyle{empty}
\newpage

%正文部分
\flushleft
\section{目的及意义}
\subsection{目的}
\paragraph{\textnormal{  提起中国白酒,不得不联想起我们的国酒——茅台。茅台,作为酱香型白酒的典范之作,以其酱香突出、香气扑鼻的特点深受国内外饮酒人士的好评。以茅台为代表的贵州名优白酒作为一张贵州旅游的名片,而逐渐为大家所熟知。那么如何充分利用这一酒文化的名片,来塑造贵州良好的旅游形象?本篇将从酒文化之于旅游事业发展的目的、意义、酒文化的发展、促进关系等多方面来论述这一观点。}}
\subsection{意义}
\paragraph{\textnormal{  自2013年起,我省经济增速就跃居全国前列,甚至达到了增速第一省份的位置。面对这样的经济转型时期,我们要把握好机遇,加速产业链的形成,以达到更好的增长效果,谋求可持续发展,规避泡沫经济。形成良好的经济增长态势,以此提高人民幸福感。
\newline   酒文化和贵州息息相关。以茅台酒为例,其历史可以追溯至公元前135年的西汉汉武帝时期\cite{history}。溯本逐源,我们有着悠久的酒文化历史,随着几千年的传承,这一制酒的古法工艺更是被发挥到极致。如果我们可以充分发挥自身优势,将旅游和我们传统的酿酒、饮酒文化相结合,必定能促进整个贵州旅游产业链的发展,实现利益最大化。提升贵州形象,也带领大家走进贵州,充分了解贵州,真正实现“醉美多彩贵州”的文化弘扬。}}
\section{酒文化的内容及其发展}
\paragraph{\textnormal{酒文化是包括酒艺、酒德、酒俗、酒文艺、酒建筑在内的与酒有关的文化体系\cite{culture}。\newline    酒文化即是物质需要与精神需要达到满足的统一。人们丰收而酿造,祭祀以求安。酒自古以来就被赋予了多重含义,成为情感的载体。例如,“浊酒一杯家万里,燕然未勒归无计”,通过杯酒之言生动地描绘出戍边将士心中的愤慨和无奈。又如,“明月几时有,把酒问青天”,每逢佳节倍思亲,古代文人普遍将这种思念寓于酒上。酒承载着人们太多强烈的情感。每感心力焦急,古人便饮酒作赋,以抒发心中的怅然之志。“酒不醉人人自醉”,酒只是一种含有酒精的饮品,它却被历代文人墨客所推崇,以致衍生出如此众多的情感,无不彰显着它源远流长的历史厚重感。人生不如意之事十之八九,每每迁谪异地,总是要取出美酒,痛饮几杯,沉沉睡去,管他这纷繁尘世,管他那昏庸无能,心中的感情,全融在这一杯酒之中,咽了,化了,随他去了。身为千里马,干的却是拉磨的活计,没有伯乐来识别我,罢了罢了,“上善若水,水善利万物而不争”,唯有这汤汤水水懂我的心,和着酒精的麻痹,让我暂时忘记这世间,只蜷缩在美好的梦里,我便无欲无求。
除了感情含义以外,酒文化还融于制酒酿酒的技艺之中。我省在1993-1997年间于仁怀修建了中国酒文化城,意在发扬这种传统的酒文化特色,传播酒文化知识。可以说,酒文化即是一种时代文化。这其中包含了政治、经济、民俗等多方面的文化。例如,我们可以从酒器之中 \cite{tool} 窥探一个朝代的文化。唐代的酒器华丽而精美,反映出这一时代特点,即是国力昌盛,饮酒之风盛行。杜甫在人生的最低谷还不忘吟一句“潦倒新停浊酒杯”,其对酒的痴迷和偏爱程度可见一斑。也反映出某些文人在酒中培养飘然若仙的审美情趣,以成佳句的时代特点。}}
\section{贵州酒文化对本土旅游业的促进作用}
\paragraph{\textnormal{  既然贵州有如此源远流长的酒历史、酒文化,那么我们怎么充分利用这一优势,使其促进我省旅游业的发展呢?\newline   贵州,作为西南的重要省份,其独特的少数民族文化一直吸引着众多游客前来观赏。可以说,贵州拥有众多的旅游资源。但是,我们是否真正地把名片打了出去?很多人提起贵州可能第一印象是黄果树,之后是茅台。那么既然茅台能作为一种贵州的旅游名片而被人们所广为熟知,我们何不扩大这一文化的影响力,借此良机提升自身形象,真正地把这张名片打出去呢?\newline    从功能方面来说,酒文化之于旅游业主要有四方面的功能,即:体验功能、娱乐功能、审美功能\cite{function}。\newline    首先,旅游,即是游客来到实地参观游玩,那么他们就自然而然地向景点提出更高的要求:要有体验的效果,让游客亲身参与其中,才能是好的旅游项目。例如,我们参观某个历史古迹时,总会情不自禁的回想起当年的一些名人轶事。站在故宫的金銮殿之内,不自然地就会回想起明清皇帝朝政的情形;站在岳阳楼,不自然地会想起范仲淹的那句“先天下之忧而忧,后天天下之乐而乐”这些都归功于场景的代入感。可以说,这种代入感在我们的酒文化背景下是比较容易建立的。它需要比较深厚的历史背景和一系列的民俗活动,让游客亲身参与其中。比如前文中提到的水族朋友们的拦门酒,这就给笔者强烈的文化代入感。\newline   其次,它还具备娱乐和审美功能。众所周知,我们外出游玩是为了让自己疲惫的身心真正放松下来,起到的应该是一种休息的作用。某些景点由于人流量大、服务不周等缺点,大大增加了游客数量在长时间内下降的可能性。我们除了必要的场景代入感,还应该兼具某些略显安静的观赏活动。例如在福建、江浙一带流行的功夫茶表演,这就给游客一种审美的享受。如果我们能够把一些酿造酒的技艺在现场重现给游客,让一些专业的品酒师举办知识普及的讲座,让游客有观赏和审美的体验,这无疑就带给了游客娱乐感和享受感.\newline    最后,酒文化对于旅游事业还有经济发展的作用。除景点门票之外,旅游本身就会带动周边产业的发展,例如住宿、餐饮等服务性行业。另外,由于我们的这种弘扬酒文化的主张,会有大批的游客愿意到名酒原产地参观,购买纪念品。这又是一大收入。过去的种种事实证明,我们不能仅仅依靠门票来发展旅游业,这种单一的旅游经济来源必然会带来一些问题,从宏观上来看,这种方式不利于周边产业的发展,过于单一,时间一长,经济增长就会疲软。从长远来看,我们更需要形成产业链,扩大和挖掘这一产业的经济潜力。形成多种服务行业、甚至第一、第二产业的结合,全方位,多层次,宽领域地促进经济发展。}}
\section{感想和总结}
\paragraph{\textnormal{\newline   笔者来自北方,之前对贵州并没有深刻的了解。这种了解可以简单地形容为一种地域特点,比如内蒙古的草原,山西的黄土高坡等。而在此求学的将近一年的时间里,虽然去的地方并不算多,我还是逐渐了解到了贵州这一赋有民族特色的文化大省。所以,笔者更习惯在此文中用“我省”来代替贵州,因为这里让我产生了一种文化的亲近感。我们常说:“民族的,就是世界的”。这种民族文化无疑是我国甚至世界各民族文化之中闪耀的一支。\newline   自1915年万国博览会至今,已有一百又一年,自茅台酒走向世界也有一百多个年头了。茅台真正吸引大众的,恐怕不只是它的酱香风味。对于不是那么了解风味的普通民众来说,打动他们的,是那份几千年坚持古法工艺制造的那份匠人之心。茅台1400多种风味成分中,少了哪一道工序,都会使味道大打折扣,是很难在后期通过勾调出来的。笔者虽没能有机会品尝其中的芳香,但那份工人们对酿酒的传承和文化的沉淀却已在心中留下了独特的,难以抹去的馥郁之香。这就是匠心独具的魅力。\newline   在中国几千年的灿烂文明中,酒文化确是其中独特的一种文化。不论是民俗活动,还是酒文化对政治经济的影响作用,无疑是劳动人民智慧的结晶。“何以解忧,唯有杜康。”酒文化担得起这份几千年文化的担子,更经得起未来几千年的考验。酒,确能载得动许多愁!\newline   (学生才疏学浅,对酒文化及贵州旅游发展虽有自己的看法,但总体上来说还是知之甚少。如有不当之处,还望先生雅正。)
}}

\clearpage
%参考文献
\begin{thebibliography}{5}
    \zihao{-4}
    \bibitem{history} \zihao{-4}\textnormal{吴慧群,王仕佐. 利用“茅台酒”品牌优势 打造中国第一酒镇——关于茅台镇旅游业发展的思考[J]. 酿酒科技,2007,09:115-119.}
    \bibitem{culture} \zihao{-4}\textnormal{陈少波. 酒文化与贵州旅游[J]. 贵阳师专学报(社会科学版),1995,03:81-84.}
    \bibitem{function}\zihao{-4}\textnormal{杨之昉.从酒器谈唐代饮酒习俗[J].文博,2005,03:38-45.}
    \bibitem{react} \zihao{-4}\textnormal{吴正光. 贵州酒文化[J]. 当代贵州,2000,03:40-42.}
    \bibitem{history} \zihao{-4}\textnormal{王仕佐,邓咏梅,黄平. 略论贵州酒文化及旅游功能[J]. 酿酒科技,2003,05:84-87.}
\end{thebibliography}

\end{document}
